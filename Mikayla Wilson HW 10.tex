%%%%%%%%%%%%%%%%%%%%%%%%%%%%%%%%%%%%%%%%%%%%%%%%%%%%%%%%%%%%
%%%%%%%%%%%%%%%%%%%%%%%%%%%%%%%%%%%%%%%%%%%%%%%%%%%%%%%%%%%%
%%%%%%%%%%%%%%%%%%%%%%%%%%%%%%%%%%%%%%%%%%%%%%%%%%%%%%%%%%%%
%%%%%%%%%%%%%%%%%%%%%%%%%%%%%%%%%%%%%%%%%%%%%%%%%%%%%%%%%%%%
%%%%%%%%%%%%%%%%%%%%%%%%%%%%%%%%%%%%%%%%%%%%%%%%%%%%%%%%%%%%
\documentclass[12pt]{article}
\usepackage{epsfig}
\usepackage{times}
\usepackage{amsmath}
\renewcommand{\topfraction}{1.0}
\renewcommand{\bottomfraction}{1.0}
\renewcommand{\textfraction}{0.0}
\setlength {\textwidth}{6.6in}
\hoffset=-1.0in
\oddsidemargin=1.00in
\marginparsep=0.0in
\marginparwidth=0.0in                                                                               
\setlength {\textheight}{9.0in}
\voffset=-1.00in
\topmargin=1.0in
\headheight=0.0in
\headsep=0.00in
\footskip=0.50in                                         
%\setcounter{page}{1}
\thispagestyle{empty}
\begin{document}
\def\pos{\medskip\quad}
\def\subpos{\smallskip \qquad}
\newfont{\nice}{cmr12 scaled 1250}
\newfont{\name}{cmr12 scaled 1080}
\newfont{\swell}{cmbx12 scaled 800}
%%%%%%%%%%%%%%%%%%%%%%%%%%%%%%%%%%%%%%%%%%%%%%%%%%%%%%%%%%%%
%     DO NOT CHANGE ANYTHING ABOVE THIS LINE
%%%%%%%%%%%%%%%%%%%%%%%%%%%%%%%%%%%%%%%%%%%%%%%%%%%%%%%%%%%%
%     DO NOT CHANGE ANYTHING ABOVE THIS LINE
%%%%%%%%%%%%%%%%%%%%%%%%%%%%%%%%%%%%%%%%%%%%%%%%%%%%%%%%%%%%
%     DO NOT CHANGE ANYTHING ABOVE THIS LINE
%%%%%%%%%%%%%%%%%%%%%%%%%%%%%%%%%%%%%%%%%%%%%%%%%%%%%%%%%%%%

\begin{center}
{\large
\bf PHYS 20323/60323: Fall 2019 - LaTeX Example}\\
\vskip0.4in
%%%%%%%%%%%%%%%%%%%%%%%%%%%%%%%%%%%%%%%%%%%%%%%%%%%%%%%%%%%%
%{\large LABORATORY X: Your Name Here}\\\vskip0.25in
%%%%%%%%%%%%%%%%%%%%%%%%%%%%%%%%%%%%%%%%%%%%%%%%%%%%%%%%%%%%
\end{center}
%%%%%%%%%%%%%%%%%%%%%%%%%%%%%%%%%%%%%%%%%%%%%%%%%%%%%%%%%%%%
% Section Heading
%%%%%%%%%%%%%%%%%%%%%%%%%%%%%%%%%%%%%%%%%%%%%%%%%%%%%%%%%%%%
%\noindent
\hspace*{1mm} {1. Consider a particle confined in a two-dimensional infinite square well} \\
% Basically an Authors List.
%Collaborators: ``who you worked with, if any.''\\
% Where can I get any electronic data associated with this project. 
%Data Location: ``directory where any project data/images are located''\\. %frac and matrix afterward
\begin{equation*}
  V(x,y) = \left\{
  \begin{matrix}
    \hspace*{1mm} 0 \hspace*{3mm}, \\ - \hspace*{3mm} \\ \infty \hspace*{1mm} , 
  \end{matrix} \hspace*{10mm}
%  \right), \left(
  \begin{matrix}
    0 \leq x \leq a , \hspace*{3mm}0<y< a\\
     \textit{otherwise}\\
  \end{matrix}
%  \right) 
 % \right\}
\end{equation*}

\vskip0.1in
%{\begin{equation*}
%	V(x,y) =
%	   \begin{cases}
%0 \hspace*{3mm},  &  0 \leq x \leq a , \hspace*{2mm}0<y< a\\ \hline
% \infty \hspace*{1mm} , & \hspace*{10mm} \textit{otherwise}
%\end{cases} 
%\end{equation*}

%%%%%%%%%%%%%%%%%%%%%%%%%%%%%%%%%%%%%%%%%%%%%%%%%%%%%%%%%%%%
% Section Heading
%%%%%%%%%%%%%%%%%%%%%%%%%%%%%%%%%%%%%%%%%%%%%%%%%%%%%%%%%%%%
The eigenfunctions have the form:
\begin{equation*}
%\begin{center}
	\Psi (x,y) = \frac{2}{a} \hspace*{1mm}
	\mathrm{ sin}\hspace*{1mm}(\frac{n\pi x}{a} ) \hspace*{1mm} \mathrm{sin}\hspace*{1mm} (\frac{m \pi y}{a} )
%\end{center}
\end{equation*}

with the corresponding energies being given by: 

\begin{equation*}
	E_{nm} = (n^{2} + m ^{2}) \frac{\pi^{2} \hbar^{2}}{2ma^{2}} 
\end{equation*}
\vskip0.15in

\hspace*{1mm}  (a)  \hspace*{1mm} (5 points) What are the levels of degeneracy of the five lowest energy values?

\vskip0.1in
\hspace*{1mm}  (b) \hspace*{1mm} (5 points) Consider a perturbation given by:
\begin{equation*} 
	\hat{H}'  =a^{2} \hspace*{1mm}  V_{0} \hspace*{1mm}  \delta \hspace*{1mm} (x - \frac{a}{2} ) \hspace*{1mm}  \delta \hspace*{1mm}  (y - \frac{a}{2} ) 
\end{equation*}

\hspace*{7mm} Calculate the first order correction to the ground state energy.
\vskip0.5in

\noindent 2. {\bf The following questions refer to stars in the Table below.}\\
\noindent \hspace*{3mm}  Note: There may be multiple answers. 

\vskip0.15in

%\begin{center}
\noindent \hspace*{3mm} \begin{tabular}{lccccr}\hline
\multicolumn{1}{|c|}{Name} & \multicolumn{1}{c|}{Mass} & \multicolumn{1}{c|}{Luminosity} & \multicolumn{1}{c|}{Lifetime} & \multicolumn{1}{c|}{Temperature} & \multicolumn{1}{c|}{Radius} \\\hline
\multicolumn{1}{|l|}{Zeta}   & \multicolumn{1}{c|}{60. $M_{sun}$}   &  \multicolumn{1}{c|}{10$^{6}$ $L_{sun}$}  & \multicolumn{1}{c|}{8.0 $\times$ 10$^{5}$ years} & \multicolumn{1}{c|}{  } & \multicolumn{1}{c|}{  } \\\hline
\multicolumn{1}{|l|}{Epsilon }  & \multicolumn{1}{c|}{6.0 $M_{sun}$}  &   \multicolumn{1}{c|}{10$^{3}$ $L_{sun}$}  &\multicolumn{1}{c|}{  } & \multicolumn{1}{c|}{20,000 K} &\multicolumn{1}{c|}{ }  \\\hline
\multicolumn{1}{|l|}{Delta}   & \multicolumn{1}{c|}{2.0 $M_{sun}$}   & \multicolumn{1}{c|}{ } &   \multicolumn{1}{c|}{5.0 $\times$ 10$^{8}$ years} & \multicolumn{1}{c|}{ } & \multicolumn{1}{c|}{2 $R_{sun}$} \\\hline
\multicolumn{1}{|l|}{Beta} & \multicolumn{1}{c|}{1.3 $M_{sun}$} & \multicolumn{1}{c|}{3.5 $L_{sun}$}  &\multicolumn{1}{c|}{ } &\multicolumn{1}{c|}{ } &\multicolumn{1}{c|}{ } \\\hline
\multicolumn{1}{|l|}{Alpha} & \multicolumn{1}{c|}{1.0 $M_{sun}$} & \multicolumn{1}{c|}{ } & \multicolumn{1}{c|}{ } & \multicolumn{1}{c|}{ } & \multicolumn{1}{c|}{1 $R_{sun}$}  \\\hline
\multicolumn{1}{|l|}{Gamma} & \multicolumn{1}{c|}{0.7 $M_{sun}$} & \multicolumn{1}{c|}{ } & \multicolumn{1}{c|}{4.5 $\times$ 10$^{10}$ years} & \multicolumn{1}{c|}{5000 K} & \multicolumn{1}{c|}{ }\\\hline
\end{tabular}\vskip 0.2in
%\end{center}
\vskip0.1in
%\hspace*{1mm} 
(a) \hspace*{1mm} (4 points) Which of these stars will produce a planetary nebula at the end of their life.

\vskip 0.5in
%\hspace*{1mm} 
(b) \hspace*{1mm} (4 points) Elements heavier than \textit{Carbon} will be produced in which stars.
\end{document}
